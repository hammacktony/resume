%% start of file `resume.tex'.
%% Copyright 2006-2015 Xavier Danaux (xdanaux@gmail.com).
%
% This work may be distributed and/or modified under the
% conditions of the LaTeX Project Public License version 1.3c,
% available at http://www.latex-project.org/lppl/.
%
% This work has the LPPL maintenance status `maintained'.
% 
% The Current Maintainer of this work is Tony Hammack (hammack.tony@gmail.com).

\documentclass[11pt,a4paper,sans]{moderncv}        % possible options include font size ('10pt', '11pt' and '12pt'), paper size ('a4paper', 'letterpaper', 'a5paper', 'legalpaper', 'executivepaper' and 'landscape') and font family ('sans' and 'roman')

% moderncv themes
\moderncvstyle{classic}                             % style options are 'casual' (default), 'classic', 'banking', 'oldstyle' and 'fancy'
\moderncvcolor{blue}                               % color options 'black', 'blue' (default), 'burgundy', 'green', 'grey', 'orange', 'purple' and 'red'
%\renewcommand{\familydefault}{\sfdefault}         % to set the default font; use '\sfdefault' for the default sans serif font, '\rmdefault' for the default roman one, or any tex font name
%\nopagenumbers{}                                  % uncomment to suppress automatic page numbering for CVs longer than one page

% character encoding
\usepackage[utf8]{inputenc}                       % if you are not using xelatex ou lualatex, replace by the encoding you are using
%\usepackage{CJKutf8}                              % if you need to use CJK to typeset your resume in Chinese, Japanese or Korean

% adjust the page margins
\usepackage[scale=0.75]{geometry}
%\setlength{\hintscolumnwidth}{3cm}                % if you want to change the width of the column with the dates
%\setlength{\makecvtitlenamewidth}{10cm}           % for the 'classic' style, if you want to force the width allocated to your name and avoid line breaks. be careful though, the length is normally calculated to avoid any overlap with your personal info; use this at your own typographical risks...

% personal data
\name{Tony}{Hammack}
% \title{Resume}                               % optional, remove / comment the line if not wanted
\address{}{Birmingham, AL} % optional, remove / comment the line if not wanted; the "postcode city" and "country" arguments can be omitted or provided empty
\email{hammack.tony@gmail.com}                               % optional, remove / comment the line if not wanted
\homepage{www.tonyhammack.com}                         % optional, remove / comment the line if not wanted
\social[linkedin]{hammack-tony}                        % optional, remove / comment the line if not wanted
\social[github]{hammacktony}                              % optional, remove / comment the line if not wanted
% \extrainfo{additional information}                 % optional, remove / comment the line if not wanted
% \photo[64pt][0.4pt]{picture}                       % optional, remove / comment the line if not wanted; '64pt' is the height the picture must be resized to, 0.4pt is the thickness of the frame around it (put it to 0pt for no frame) and 'picture' is the name of the picture file
% \quote{Some quote}                                 % optional, remove / comment the line if not wanted


%----------------------------------------------------------------------------------
%            content
%----------------------------------------------------------------------------------
\begin{document}

%-----       resume       ---------------------------------------------------------
\makecvtitle

%%% Import contents
%% education.tex
%% Copyright 2006-2015 Xavier Danaux (xdanaux@gmail.com).
%
% This work may be distributed and/or modified under the
% conditions of the LaTeX Project Public License version 1.3c,
% available at http://www.latex-project.org/lppl/.
%
% This work has the LPPL maintenance status `maintained'.
% 
% The Current Maintainer of this work is Tony Hammack (hammack.tony@gmail.com).

\section{Education}
\cventry
{2016--2018}
{M.S. in Mathematics}
{University of Alabama at Birmingham}
{Birmingham, AL}
{\textit{GPA: \textbf{4.0} (4.0 scale)}}
{Focused on applied mathematics, linear algebra, numerical linear algebra, statistics, and partial differential equations.}  % arguments 3 to 6 can be left empty

\cventry{2012 - 2016}
{B.S. in Mathematics}
{University of Mobile}
{Mobile, AL}
{\textit{GPA: \textbf{3.96} (4.0 scale)}}
{Minored in Chemistry}

%% Copyright 2006-2015 Xavier Danaux (xdanaux@gmail.com).
%
% This work may be distributed and/or modified under the
% conditions of the LaTeX Project Public License version 1.3c,
% available at http://www.latex-project.org/lppl/.
%
% This work has the LPPL maintenance status `maintained'.
% 
% The Current Maintainer of this work is Tony Hammack (hammack.tony@gmail.com).


\section{Experience}


% Job 1
\cventry{2020 - Present}
{Machine Learning Engineer}
{ProctorU}
{Birmingham, AL}{}
{
\begin{itemize}%
\item Decoupled old machine learning solution from main monolith
\item Implemented modular computer vision machine learning solutions
\end{itemize}}


% Job 2
\cventry{2018 - 2020}
{Machine Learning Engineer}
{Positronic AI}
{St. Louis, MO}{}
{
\begin{itemize}%
\item Helped implement a computer vision information retrieval system
\item Developed models for action recognition tasks
\item Developed an in-house machine learning tool to help foster productivity among other machine learning engineers
\end{itemize}}


% Job 3
\cventry{2016 - 2018}
{Graduate Assistant}
{University of Alabama in Birmingham}
{Birmingham, AL}{}
{
\begin{itemize}%
\item Taught multiple Pre-Calculus classes
\item Helped run the Mathematics Learning Lab
\end{itemize}}


%% Copyright 2006-2015 Xavier Danaux (xdanaux@gmail.com).
%
% This work may be distributed and/or modified under the
% conditions of the LaTeX Project Public License version 1.3c,
% available at http://www.latex-project.org/lppl/.
%
% This work has the LPPL maintenance status `maintained'.
% 
% The Current Maintainer of this work is Tony Hammack (hammack.tony@gmail.com).


\section{Skills}
\cvdoubleitem{}{Python}{}{}
\cvdoubleitem{}{Javascript}{}{}
\cvdoubleitem{}{SQL}{}{}
\cvdoubleitem{}{MongoDB}{}{}
\cvdoubleitem{}{Ruby}{}{}








\end{document}
%% end of file `resume.tex'.
